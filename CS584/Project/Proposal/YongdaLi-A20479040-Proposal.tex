\documentclass{article}

\title{Cassava Leaf Disease Classification}
\author{Yongda Li}
\date{February 2021}

\begin{document}
\maketitle
\tableofcontents
\bibliographystyle{unsrt}

\section{Team Members}
Yongda Li A20479040

\section{Description of the Problem}

As the second-largest provider of carbohydrates in Africa, cassava is a key food security crop grown by smallholder farmers because it can withstand harsh conditions. At least 80\% of household farms in Sub-Saharan Africa grow this starchy root, but viral diseases are major sources of poor yields.We can identify common diseases with the help of machine learning, so these diseases can be treated.

Existing methods of disease detection require farmers to solicit the help of government-funded agricultural experts to visually inspect and diagnose the plants. This suffers from being labor-intensive, low-supply and costly. As an added challenge, effective solutions for farmers must perform well under significant constraints, since African farmers may only have access to mobile-quality cameras with low-bandwidth.

In this problem, I introduce a dataset of 21,367 labeled images from kaggle.

What I want to do is to classify each cassava image into four disease categories or a fifth category indicating a healthy leaf. After this problem has solved, farmers may be able to quickly identify diseased plants, potentially saving their crops before they inflict irreparable damage.

\section{What have been done}
I have collected the data sets that I need to use, and have roughly conceived the machine learning training process in my mind. At the same time, I checked the four common diseases of cassava are Cassava Bacterial Blight(CBB), Cassava Brown Streek Disease(CBSD), Cassava Green Mottle(CGM) and Cassava Mosaic Disease(CMD). And inquired about the practical significance of the problem in Africa, and found that if the problem can be solved well, it will have huge benefits for agricultural production in Africa.

\section{Why it is different}
This proposal work has strong practical significance and focuses on solving the problems encountered by people in actual agricultural production. It is of great importance to help increase the yield of cassava.

\section{Preliminary Plan}
\subsection{now - 20.03.2021}
Build a simple machine learning classification algorithm to classify cassava pictures.\cite{goodfellow2016deep}

\subsection{21.03.2021 - 20.04.2021}

Use complex methods on the basis of simple algorithms, such as deep learning\cite{bishop2006pattern}, and strengthen classification algorithms to get better classification results.

\subsection{21.04.2021 - 01.05.2021}
Complete the final project report.

\bibliography{reference}


\end{document}
